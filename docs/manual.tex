\documentclass[11pt]{report}

\usepackage{times,fullpage,graphicx,amsmath}
\usepackage[pdfborder={0 0 0}]{hyperref}

\usepackage{subfigure}

\newcommand{\supth}{{\mathrm{th}}}


\title{LibraryComplexity Manual}
\author{Timothy Daley \and Andrew Smith}

\begin{document}
\maketitle

\chapter{Quick Start}
\label{chap:quickstart}

The LibraryComplexity package is aimed at predicting
the yield and number of distinct reads from a genomic library
from an initial sequencing experiment.  The estimates
can then be used to examine the utility of further
sequencing, optimize the sequencing depth,
or to screen multiple libraries to avoid low complexity
samples.

\section{Installation}
\label{sec:install}

\subsection{Download}
\label{sub:download}

LibraryComplexity is availible at
\url{http://smithlab.cmb.usc.edu/software/librarycomplexity/}.

\subsection{System Requirements}
\label{sub:require}

LibraryComplexity runs on Unix-type system
with GNU Scientific Library (GSL) and
GNU Complilation Collection (GCC) (if you would like to compile it
yourself).  It has been tested on Linux and 
Mac OS-X.

\subsection{Installation}
\label{sub:install}

Download the source code and decompress
it with 
\begin{verbatim}
 $ tar xvfz library_complexity.tar.gz
\end{verbatim}
% 
Enter the library\_complexity directory and run
\begin{verbatim}
$ make all
\end{verbatim}
If the desired input files are in .bam format, bamtools is required, 
available at \url{https://github.com/pezmaster31/bamtools}.  
To compile, the bamtools directory must be specified, i.e.
\begin{verbatim}
$ make all BAMTOOLS_ROOT=~/bamtools/
\end{verbatim}
If compiled successfully, the executable files are available
in \textbf{library\_complexity/}.

\section{Using LibraryComplexity}
\label{sec:usage}

\subsection{Basic usage:}
\label{sub:basic}

To generate the complexity plot of a genomic
library from a sorted read file in .bed or .bam format,
use the program \textit{complexity\_plot}.  Use
-o to specify the output name.

\begin{verbatim}
$ complexity_plot -o output.txt input.bed
\end{verbatim}

To estimate the future yield 
and bounds on the number of distinct reads
of a genomic library
using an initial experiment in .bed or .bam format,
use the program \textit{library\_complexity}.
Use -o to specify the output of the yield
estimates and -L to specify the output of
the bounds.  -v will print more information
and will print the library size bounds, if -L is 
omitted.  The options
-e and -s set the maximum number of 
total reads from which yield estimates are desired
and the number of reads between estimates, respectively.
For confidence intervals of the estimates, -b controls
the number of resamples to take and -a controls the confidence
level.  The last parameter is a .bed or .bam
file sorted by chromosome and genomic position.


\begin{verbatim}
$ library_complexity -o yield.txt -L size_bounds.txt input.txt
\end{verbatim}

\section{File Format}
\label{sec:format}

Input files are mapped read files sorted by
chromosome and position in either .bed or .bam 
format.  If files are in .bam format, bamtools is required
prior to installation, as detailed in the Installation 
section~\ref{sub:install}.

\chapter{Detailed usage}
\label{chap:detail}

\section{complexity\_plot}
\label{sec:complexityplot}

complexity\_plot is used to compute the 
expected complexity curve of a mapped read file by 
subsampling without replacement 
and counting the distinct reads.

\begin{description}
\item[-o, -output] Name of output file, default prints to screen
\item[-l, -lower] Lower limit of sampling, default is 1M reads
\item[-u, -upper] Upper limit of sampling, default is the number of reads
in the input file.
\item[-s, -step] Step size between sampling points, default is 1M reads.
\item[-v -verbose] Prints more information
\item[-b, -bam] Input file is in .bam format.  The program must be compiled with bamtools, see Installation~\ref{sub:install}.
\end{description}

\section{library\_complexity}
\label{sec:librarycomplexity}

library\_complexity is used to compute
the expected yield for theoretical larger
experiments and bounds on the number of distinct 
reads in the library.

\begin{description}
\item[-o, -output








\end{document}