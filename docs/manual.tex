\documentclass[11pt, titlepage]{article}

\usepackage{times,fullpage,graphicx,amsmath, cite, alltt, enumitem, titlesec, verbatim, siunitx}
\usepackage[pdfborder={0 0 0}]{hyperref}
\usepackage{lipsum}
\newcounter{question}
\setcounter{question}{0}

\usepackage{subfigure}

\newcommand\Que[1]{%
   \leavevmode\par
   \stepcounter{question}
   \noindent
   \thequestion. Q --- #1\par}

\newcommand\Ans[1]{%
    \leavevmode\par\noindent
   {\leftskip37pt
    A. --- #1 \par }}

\newcommand{\fn}[1]{\texttt{#1}}


\newcommand{\supth}{{\mathrm{th}}}
\newcommand{\tab}{\hspace*{2em}}

\titleformat*{\paragraph}{\large\bfseries}

\title{The \textbf{preseq} Manual}
\author{Timothy Daley \and Victoria Helus \and Chao Deng \and Andrew Smith }

\begin{document}
\maketitle

\tableofcontents

\newpage

\section{Quick Start}
\label{chap:quickstart}

The \textbf{preseq} package is aimed to help researchers design and
optimize sequencing experiments by using population sampling models to
infer properties of the population or the behavior under deeper
sampling based upon a small initial sequencing experiment.  The
estimates can then be used to examine the utility of further
sequencing, optimize the sequencing depth, or to screen multiple
libraries to avoid low complexity samples.

The four main programs are \fn{c\_curve}, \fn{lc\_extrap},
\fn{gc\_extrap}, and \fn{bound\_pop}.  \fn{c\_curve} interpolates the
expected complexity curve based upon a hypergeometric formula and is
primarily used to check predictions from \fn{lc\_extrap} and
\fn{gc\_extrap}.  \fn{lc\_extrap} uses rational function
approximations of Good \& Toulmin's~\cite{good1956number}
non-parametric empirical Bayes estimator to predict the library
complexity of future experiments, in essence looking into the future
for hypothetical experiments.

\fn{gc\_extrap} uses a similar approach as \fn{lc\_extrap} to predict
the genome coverage, i.e. the number of bases covered at least once,
from deeper sequencing in a single cell or low input sequencing
experiment based on the observed coverage counts.  An option is
available to predict the coverage based on binned coverage counts to
speed up the estimates.  \fn{gc\_extrap} requires mapped read or bed
format input, so the tool \fn{to-mr} is provided to convert bam format
read to mapped read format.

\fn{bound\_pop} uses a non-parametric moment-based approach to
conservatively estimate the total number of classes in the sample,
also called the species richness of the population that is sampled.

\newpage

\section{Installation}
\label{sec:install}

\paragraph{Download}
\label{sub:download}~\\~\\[-.2cm]
\raggedright{\textbf{preseq} is available at }
\url{http://smithlabresearch.org/software/preseq/}
or \url{https://github.com/smithlabcode/preseq}.


\paragraph{System Requirements}
\label{sub:require}
~\\~\\[-.2cm]
\textbf{preseq} runs on Unix-type system
with GNU Scientific Library (GSL), available
at ~\url{http://www.gnu.org/software/gsl/}.
If the input file is in BAM format, the HTSLib
API is required but is included in all binaries and
source code.
If the input is a text file of counts in a single column or is
in BED format,
SAMTools is not required.
It has been tested on Linux and
Mac OS-X.

\paragraph{Installation}~\\~\\[-.2cm]
\label{sub:install}
If the source code was downloaded from the Smithlab
website the first step is to decompress it using the
command
\begingroup \fontsize{9pt}{12pt}\selectfont \begin{alltt}
 $ tar -jxvf preseq.tar.bz2
\end{alltt} \endgroup
To download the source code from GitHub, use
the command
\begingroup \fontsize{9pt}{12pt}\selectfont \begin{alltt}
 $ git clone --recursive git://github.com/smithlabcode/preseq.git
\end{alltt} \endgroup
%
In both cases, enter the \textbf{preseq/} directory and run
\begingroup \fontsize{9pt}{12pt}\selectfont \begin{alltt}
$ make all
\end{alltt}\endgroup
to compile all the code.

If one wishes to link to HTSLib API not
included with the source code, it should be installed
system-wide or the preseq Makefile will need modifying.
If HTSLib is available on the system, then instead run
\begingroup \fontsize{9pt}{12pt}\selectfont \begin{alltt}
$ make HAVE_HTSLIB=1 all
\end{alltt}\endgroup
Output after typing this command should include the flag
\fn{-DHAVE\_SAMTOOLS} if the linking is successful. If compiled
successfully, the executable file is available in \textbf{preseq/}.

If a BAM file is used as input without successful linking to
HTSLib, then the following error will occur:
\begingroup \fontsize{9pt}{12pt}\selectfont \fn{terminate called after throwing an instance of 'std::string'}\endgroup.

\newpage

\section{Using \textbf{preseq}}
\label{sec:usage}

\paragraph{Basic usage}~\\~\\[-.2cm]
\label{sub:basic}
To generate the complexity curve of a genomic
library from a read file in BED or BAM format or a duplicate count file,
use the function \fn{c\_curve}.  Use
\fn{-o} to specify the output name.
\begingroup \fontsize{9pt}{12pt}\selectfont \begin{alltt}
$ ./preseq c_curve -o complexity_output.txt input.bed
\end{alltt}\endgroup

To predict the complexity curve
of a sequencing library
using an initial experiment in BED format,
use the function \fn{lc\_extrap}.
The required options are \fn{-o} to specify
the output of the yield estimates and
the input file, which is either a BED
file sorted by chromosome, start position, end
position, and strand or a BAM file
sorted with the samtools sort function. Additional
options are available and are detailed below.
\begingroup \fontsize{9pt}{12pt}\selectfont
\begin{alltt}
$ ./preseq lc_extrap -o future_yield.txt input.bed
\end{alltt}\endgroup

For a low input sequencing experiment the genomic
coverage is highly variable and uncertain function
of sequencing depth.  Some regions may be missing
due to locus dropout or preferentially amplified during
whole genome amplification.
\fn{gc\_extrap} allows the level genomic coverage from deep
sequencing to be predicted based on an initial sample.
The input file format need to be a mapped read (MR) or BED,
sorted by chromosome, start position, and end position.
Additional
options are available and are detailed below.
\begingroup \fontsize{9pt}{12pt}\selectfont
\begin{alltt}
$ ./preseq gc_extrap -o future_coverage.txt input.mr
\end{alltt}\endgroup


\newpage

\section{File Format}
\label{sec:format}

\paragraph{Sorted read files in BED or BAM format}~\\~\\[-.2cm]
Input files are sorted mapped read files in BED or BAM format,
or a text file consisting of one column giving the observed read counts.
The
programs require that BED files are sorted by chromosome,
start position, end position, and strand.  This can be achieved
by using the command line function sort as follows:
\begingroup \fontsize{9pt}{12pt}\selectfont \begin{alltt}
sort -k 1,1 -k 2,2n -k 3,3n -k 6,6 input.bed > input.sort.bed
\end{alltt}\endgroup
BAM format read files should be sorted by chromosome and
start position.  This can be done with the SAMTools
sort function. If the input is in BAM format, then the flag
\fn{-B} must be included.


If the input is paired end, the option \fn{-P} can be set.
In this case concordantly mapped reads and disconcordantly
mapped fragments are counted.  This means that both ends
of a disconcordantly mapped read will each be counted separately.
If a large number of reads are disconcordant, then
the default single end should be used or the disconcordantly
mapped reads removed prior to running \textbf{preseq}.
In this case only the mapping
location of the first mate
is used as the unique molecular identifier~\cite{kivioja2011counting}.



\paragraph{Text files of observed read counts}~\\~\\[-.2cm]
For more general applications \textbf{preseq} allows the input
to be a text file of observed read counts, one count per
line.   To specify this input, the option \fn{-V} must be set.

Such a text file can typically be constructed by command
line arguments.  Take for example an unmapped sequencing
experiment in FASTQ format.  To predict the complexity, the unique
molecular identifier needs to use only the observed sequence.
For instance, a unique molecular identifier used may be the first
20 bases in the observed sequence.  A command line argument
to construct the counts would then be
\begingroup \fontsize{9pt}{12pt}\selectfont \begin{alltt}
awk '{\{ if (NR\%4==2) print substr($0,1,20); \}}' input.fastq | sort | uniq -c
\end{alltt}\endgroup
\begingroup \fontsize{9pt}{12pt}\selectfont \begin{alltt}| awk '{\{ print $1 \}}' > counts.txt
\end{alltt}\endgroup
More complicated unique molecular identifiers
can be used, such as mapping position plus a random barcode,
but are too complicated to detail in this manual.
For questions with such usage, please contact us at
\href{mailto:tdaley@usc.edu}{\nolinkurl{tdaley@usc.edu}}

\paragraph{Mapped read format for \fn{gc\_extrap}}~\\~\\[-.2cm]


\fn{gc\_extrap} does not allow for input files to be in BAM format.
We have found that some mappers give inconsistent SAM flags
for paired end reads, preventing efficient merging of reads in
the proper order.
We provide the tool \fn{to-mr}
to convert SAM or BAM format files to MR format.  The MR
or BED format file needs to be sorted by chromosome, start, and
end position before input into \fn{gc\_extrap}.


\newpage

\section{Detailed usage}
\label{chap:detail}

\paragraph{c\_curve}~\\~\\[-.2cm]
\label{sec:complexityplot}

\fn{c\_curve} is used to compute the
expected complexity curve of a mapped read file
with a hypergeometric formula~\cite{heck1975explicit}.
Output is a text file with two
columns.  The first gives the total number
of reads and the second the corresponding number
of distinct reads.

\begin{description}[style=multiline,leftmargin=6cm,font=\ttfamily]
\item[\begingroup \fontsize{9pt}{12pt}\selectfont-o, -output\endgroup] Name of output file. Default prints to screen
\item[\begingroup \fontsize{9pt}{12pt}\selectfont-s, -step\endgroup] The step size for samples. Default is 1 million reads
\item[\begingroup \fontsize{9pt}{12pt}\selectfont-v -verbose\endgroup] Prints more information
\item[\begingroup \fontsize{9pt}{12pt}\selectfont-B, -bam\endgroup] Input file is in BAM format
\item[\begingroup \fontsize{9pt}{12pt}\selectfont-P, -pe\endgroup] Input is a paired end read file
\item[\begingroup \fontsize{9pt}{12pt}\selectfont-H, -hist\endgroup] Input is a text file of the observed histogram
\item[\begingroup \fontsize{9pt}{12pt}\selectfont-V, -vals\endgroup] Input is a text file of read counts
\end{description}

\newpage

\paragraph{lc\_extrap}~\\~\\[-.2cm]
\label{sec:librarycomplexity}

\fn{lc\_extrap} is used to generate
the expected yield for theoretical larger
experiments and bounds on the number of distinct
reads in the library and the associated confidence
intervals, which is computed by bootstrapping the observed duplicate counts histogram.
Output is a text file with four columns.  The
first is the total number of reads, second
gives the corresponding average
expected number of distinct reads, and the
third and fourth give the lower and
upper
limits of the confidence interval.
Specifying verbose will print out the counts histogram
of the input file.


\begin{description}[style=multiline,leftmargin=6cm,font=\ttfamily]
\item[\begingroup \fontsize{9pt}{12pt}\selectfont-o, -output\endgroup] Name of output file. Default prints to screen
\item[\begingroup \fontsize{9pt}{12pt}\selectfont-e, -extrap\endgroup] Max extrapolation. Default is \num{1e10}
\item[\begingroup \fontsize{9pt}{12pt}\selectfont-s, -step\endgroup] The step size for samples. Default is 1 million reads
\item[\begingroup \fontsize{9pt}{12pt}\selectfont-n, -bootstraps\endgroup] The number of bootstraps. Default is 100
\item[\begingroup \fontsize{9pt}{12pt}\selectfont-c, -cval\endgroup] Level for confidence intervals. Default is 0.95
\item[\begingroup \fontsize{9pt}{12pt}\selectfont-x, -terms\endgroup] Max number of terms for extrapolation. Default is 100
\item[\begingroup \fontsize{9pt}{12pt}\selectfont-v -verbose\endgroup] Prints more information
\item[\begingroup \fontsize{9pt}{12pt}\selectfont-B, -bam\endgroup] Input file is in BAM format
\item[\begingroup \fontsize{9pt}{12pt}\selectfont-P, -pe\endgroup] Input is a paired end read file
\item[\begingroup \fontsize{9pt}{12pt}\selectfont-H, -hist\endgroup] Input is a text file of the observed histogram
\item[\begingroup \fontsize{9pt}{12pt}\selectfont-V, -vals\endgroup] Input is a text file of read counts
\item[\begingroup \fontsize{9pt}{12pt}\selectfont-Q, -quick\endgroup] Quick mode, option to estimate yield without bootstrapping for confidence intervals
\item[\begingroup \fontsize{9pt}{12pt}\selectfont-D, -defects\endgroup] Defects mode, estimates the complexity curve without checking for instabilities in the curve.  Should only be used on datasets that fail estimation without defects.
\end{description}

\newpage

\paragraph{gc\_extrap}~\\~\\[-.2cm]
\label{sec:genomiccoverage}

For single cell or low input sequencing experiments
\fn{gc\_extrap} is used to extrapolate
the expected number of bases covered at least once
for theoretical larger experiments.
Input format is required to be in mapped read
format and we have provided the tool
\fn{to-mr} to convert bam format files to mr.
Output is a text file with four columns.  The
first is the total number of sequenced and mapped
bases, second gives the corresponding
expected number of distinct bases covered, and the
third and fourth give the lower and
upper limits of the confidence interval.
Specifying verbose will print out the coverage counts histogram
of the input file.


\begin{description}[style=multiline,leftmargin=6cm,font=\ttfamily]
\item[\begingroup \fontsize{9pt}{12pt}\selectfont-o, -output\endgroup] Name of output file. Default prints to screen
\item[\begingroup \fontsize{9pt}{12pt}\selectfont-w, -max\_width\endgroup] max fragment length, set equal to read length for single end reads
\item[\begingroup \fontsize{9pt}{12pt}\selectfont-b, -bin\_size\endgroup] bin size.  Default is 10
\item[\begingroup \fontsize{9pt}{12pt}\selectfont-e, -extrap\endgroup] Maximum extrapolation in base pairs. Default is \num{1e12}
\item[\begingroup \fontsize{9pt}{12pt}\selectfont-s, -step\endgroup] The step size in bases between extrapolation points. Default is 100 million base pairs
\item[\begingroup \fontsize{9pt}{12pt}\selectfont-n, -bootstraps\endgroup] The number of bootstraps. Default is 100
\item[\begingroup \fontsize{9pt}{12pt}\selectfont-c, -cval\endgroup] Level for confidence intervals. Default is 0.95
\item[\begingroup \fontsize{9pt}{12pt}\selectfont-x, -terms\endgroup] Max number of terms for extrapolation. Default is 100
\item[\begingroup \fontsize{9pt}{12pt}\selectfont-v -verbose\endgroup] Prints more information
\item[\begingroup \fontsize{9pt}{12pt}\selectfont-D, -bed\endgroup] Input file is in BED format without sequence information
\item[\begingroup \fontsize{9pt}{12pt}\selectfont-Q, -quick\endgroup] Quick mode, option to estimate genomic coverage without bootstrapping for confidence intervals
\end{description}

\newpage

\paragraph{bound\_pop}~\\~\\[-.2cm]
\label{sec:lib_size}

\fn{bound\_pop} is a method for estimating species
richness, the total number of species or classes
in the sampled population.
Input format is the same as \fn{lc\_extrap}.
Default output is a three column text file,
with the first column containing the estimated
species richness and the second and third
containing the estimated lower and upper
confidence intervals.
If \fn{bound\_pop} is run in quick mode, then
the output is two columns.
The first column will contain the estimated species
richness and the second column will contain
the dimension or order of the approximation.



\begin{description}[style=multiline,leftmargin=6cm,font=\ttfamily]
\item[\begingroup \fontsize{9pt}{12pt}\selectfont-o, -output\endgroup] Name of output file. Default prints to screen
\item[\begingroup \fontsize{9pt}{12pt}\selectfont-p, -max\_num\_points\endgroup] Maximum number of points to use in the quadrature estimator.  Default is 10, corresponding to 20 entries of the counts histogram being used.
\item[\begingroup \fontsize{9pt}{12pt}\selectfont-t, -tolerance\endgroup] Numerical tolerance for convergence of QR algorithm.  Default is $1\mbox{\sc{e}}-20$.
\item[\begingroup \fontsize{9pt}{12pt}\selectfont-n, -bootstraps\endgroup] The number of bootstraps. Default is 100.
\item[\begingroup \fontsize{9pt}{12pt}\selectfont-c, -clevel\endgroup] Level for confidence intervals. Default is 0.95.
\item[\begingroup \fontsize{9pt}{12pt}\selectfont-v -verbose\endgroup] Prints more information.
\item[\begingroup \fontsize{9pt}{12pt}\selectfont-B, -bam\endgroup] Input file is in BAM format.
\item[\begingroup \fontsize{9pt}{12pt}\selectfont-P, -pe\endgroup] Input is a paired end read file.
\item[\begingroup \fontsize{9pt}{12pt}\selectfont-H, -hist\endgroup] Input is a text file of the observed histogram.
\item[\begingroup \fontsize{9pt}{12pt}\selectfont-V, -vals\endgroup] Input is a text file of read counts.
\item[\begingroup \fontsize{9pt}{12pt}\selectfont-Q, -quick\endgroup] Quick mode, option to estimate species richness without bootstrapping for confidence intervals.
\end{description}


\newpage

\section{lc\_extrap Examples}
\label{sec:examples}

Usage and output of \fn{c\_curve} is similar, so the following examples are of \fn{lc\_extrap} and its different options.


\paragraph{Using a sorted read file in BED (or BAM with the \fn{-B} flag) format as input}~\\
\begingroup \fontsize{9pt}{12pt}\selectfont \begin{alltt}
$ ./preseq lc_extrap -o future_yield.txt input.bed
\end{alltt}\endgroup

\begin{table}[ht!]
 \fontfamily{pcr}\fontsize{9pt}{12pt}\selectfont
\begin{tabular}{llll}

TOTAL\_READS &  EXPECTED\_DISTINCT  & LOGNORMAL\_LOWER\_95\%CI  & LOGNORMAL\_UPPER\_95\%CI \\ \hline
0 & 0 & 0 & 0\\
1000000.0 & 955978.6 & 953946.4 & 958015.1\\
2000000.0 & 1897632.0 & 1892888.4 & 1902387.5\\
3000000.0 & 2829410.5 & 2819146.4 & 2839712.0\\
4000000.0 & 3751924.0 & 3732334.5 & 3771616.2\\
. & . & . & .\\
. & . & . & .\\
. & . & . & .\\
. & . & . & .\\
9999000000.0 & 185394069.4 & 76262245.8 & 450694319.0\\
\end{tabular}
\end{table}



This example uses a sorted read file in BED format from an initial experiment generated from single sperm cells.
As noted above, the default step size between yield estimates is 1 million, the default confidence interval level is 95\%, and the default extrapolation length is 10 billion.

\paragraph{Using a sorted read file in BED format as input, including the verbose option}~\\
\begingroup \fontsize{9pt}{12pt}\selectfont \begin{alltt}
$ ./preseq lc_extrap -o future_yield.txt input.bed -v
\end{alltt}\endgroup

As \fn{lc\_extrap} is running, information will print to screen that gives a read counts histogram of the input file which truncates after the first bin value that has zero observations. Included here is the first 10 lines of what would be observed:


\begingroup \fontsize{9pt}{12pt}\selectfont \begin{alltt}
TOTAL READS     = 536855
DISTINCT READS  = 516200
DISTINCT COUNTS = 48
MAX COUNT       = 269
COUNTS OF 1     = 511413
MAX TERMS       = 100
OBSERVED COUNTS (270)
1       511413
2       2202
3       597
.
.
\end{alltt}\endgroup

\newpage
\paragraph{Using a sorted read file in BED format as input, with options}~\\
\begingroup \fontsize{9pt}{10pt}\selectfont \begin{alltt}
$ ./preseq lc_extrap -e 15000000 -s 500000 -b 90 -c .90 -o future_yield.txt input.bed
\end{alltt}\endgroup

\begin{table}[ht!]
 \fontfamily{pcr}\fontsize{9pt}{12pt}\selectfont
\begin{tabular}{llll}
TOTAL\_READS & EXPECTED\_DISTINCT & LOGNORMAL\_LOWER\_90\%CI & LOGNORMAL\_UPPER\_90\%CI \\ \hline
0 & 0 & 0 & 0 \\
500000.0 & 481098.5 & 480329.1 & 481869.1 \\
1000000.0 & 956070.6 & 954493.7 & 957650.2 \\
1500000.0 & 1428183.4 & 1425461.7 & 1430910.2 \\
2000000.0 & 1897886.0 & 1892501.7 & 1903285.7 \\
. & . & . & .\\
. & . & . & .\\
. & . & . & .\\
. & . & . & .\\
14500000.0 & 12932529.0 & 12056525.8 & 13872180.9\\
\end{tabular}
\end{table}

Notice the slight changes, with the step sizes of the extrapolation now at 500,000 as specified, and the maximum extrapolations ending at 15,000,000. The confidence intervals are now at a level of 90\%.

\paragraph{Using a histogram or read counts as input}~\\~\\[-.2cm]

\fn{lc\_extrap} allows the input file to be an observed histogram. An example of the format of this histogram is as followed:
\begingroup \fontsize{9pt}{12pt}\selectfont \begin{alltt}
1       1.68166e+07
2       4.55019e+06
3       1.93787e+06
4       1.07257e+06
5       708034
6       513134
7       384077
8       282560
9       206108
10      146334
\end{alltt}\endgroup

The following command will give output of the same format as the above examples.
\begingroup \fontsize{9pt}{12pt}\selectfont \begin{alltt}
$./preseq lc_extrap -o future_yield.txt -H histogram.txt
\end{alltt}\endgroup

Similarly, both \fn{lc\_extrap} and \fn{c\_curve} allow the option to input read counts (text file should contain ONLY the observed counts in a single column). For example, if a dataset had the following counts histogram:

\begingroup \fontsize{9pt}{12pt}\selectfont \begin{alltt}
1      4
2      3
3      1
\end{alltt}\endgroup
\newpage
Then, the corresponding input file of just read counts could be as such:
\begingroup \fontsize{9pt}{12pt}\selectfont \begin{alltt}
1
2
1
1
3
2
2
1
\end{alltt}\endgroup

Command should be run with the \fn{-V} flag (not to be confused with \fn{-v} for verbose mode):
\begingroup \fontsize{9pt}{12pt}\selectfont \begin{alltt}
$./preseq lc_extrap -o future_yield.txt -V counts.txt
\end{alltt}\endgroup

\newpage

\section{gc\_extrap Example}

\fn{gc\_extrap} is designed for coverage extrapolation
in single cell whole genome sequencing experiments.
For illustrative purposes we will examine an MDA whole
genome sequencing experiment, SRA accession
SRR1777281.  This experiment has 5.76 million paired end
101 base pair reads.
We mapped the experiment with bowtie2 v0.0-beta7
under default parameters.  This resulted in 3.63 million
concordantly mapped fragment pairs and 2.1 million
disconcordantly mapped fragments.

The first step is to convert the sorted bam file to mr format
and sort it.
\begingroup \fontsize{9pt}{12pt}\selectfont \begin{alltt}
$ ./to-mr -o SRR1777281_bwt2.mr -L 10000 SRR1777281_bwt2.sort.bam
$ sort -k 1,1 -k 2,2n -k 3,3n SRR1777281_bwt2.mr > SRR1777281_bwt2.sort.mr
\end{alltt}\endgroup
The resulting mapped read has 813 million bases (note
that bases covered by two fragments of the same read are
only counted once) and 410 million covered bases.

As a default, \fn{gc\_extrap} divides the genome into
10 base pair non-overlapping bins.
In default mode, the running time of \fn{gc\_extrap}
was under 12 minutes.
\begingroup \fontsize{9pt}{12pt}\selectfont \begin{alltt}
LOADING READS
MAPPED READ FORMAT
TOTAL READS         = 5726883
BASE STEP SIZE      = 1e+08
BIN STEP SIZE       = 1e+07
TOTAL BINS          = 8.1325e+07
BINS PER READ       = 14.2006
DISTINCT BINS       = 4.09582e+07
TOTAL BASES         = 8.1325e+08
TOTAL COVERED BASES = 4.09582e+08
MAX COVERAGE COUNT  = 79775
COUNTS OF 1         = 3.13723e+07
OBSERVED BIN COUNTS (79776)
1       3.13723e+07
2       6.97799e+06
3       1.72101e+06
.
.
\end{alltt}\endgroup

The output is as follows:
\begin{table}[ht!]
 \fontfamily{pcr}\fontsize{9pt}{12pt}\selectfont
\begin{tabular}{llll}
TOTAL\_BASES & EXPECTED\_COVERED\_BASES & LOWER\_95\%CI & UPPER\_95\%CI \\ \hline
0& 0 & 0 & 0 \\
100000000.0 & 64522380.0 & 63075879.1 & 66002053.1 \\
200000000.0 & 123422455.0 & 120747705.7 & 126156454.1 \\
300000000.0 & 178054120.0 & 174319619.1 & 181868626.2 \\
400000000.0 & 229008295.0 & 224352188.5 & 233761032.3 \\
500000000.0 & 276727080.0 & 271265011.6 & 282299130.1 \\
. & . & . & .\\
. & . & . & .\\
999900000000.0 & 1826891418.6 & 1621208958.1 & 2058668772.4 \\
\end{tabular}
\end{table}

To run \fn{gc\_extrap} at single base resolution, the option
\fn{-b 1} is required.  This results in a significant increase
in the running time of the algorithm.  For this case the
running time was 113 minutes.
\begingroup \fontsize{9pt}{12pt}\selectfont \begin{alltt}
$ ./preseq gc_extrap SRR1777281_bwt2.sort.mr -o SRR1777281_bwt2_1bp_gc_extrap.txt -b 1 -v
\end{alltt}\endgroup

\begingroup \fontsize{9pt}{12pt}\selectfont \begin{alltt}
LOADING READS
MAPPED READ FORMAT
TOTAL READS         = 5726883
BASE STEP SIZE      = 1e+08
BIN STEP SIZE       = 1e+08
TOTAL BINS          = 8.13236e+08
BINS PER READ       = 142.003
DISTINCT BINS       = 4.09454e+08
TOTAL BASES         = 8.13236e+08
TOTAL COVERED BASES = 4.09454e+08
MAX COVERAGE COUNT  = 80028
COUNTS OF 1         = 3.13527e+08
OBSERVED BIN COUNTS (80029)
1       3.13527e+08
2       6.98288e+07
3       1.722e+07
.
.
\end{alltt}\endgroup

\begin{table}[ht!]
 \fontfamily{pcr}\fontsize{9pt}{12pt}\selectfont
\begin{tabular}{llll}
TOTAL\_BASES & EXPECTED\_COVERED\_BASES & LOWER\_95\%CI & UPPER\_95\%CI \\ \hline
0& 0 & 0 & 0 \\
100000000.0 & 64680427.0 & 64284593.4 & 65078698.0 \\
200000000.0 & 123709880.0 & 122978429.6 & 124445680.9 \\
300000000.0 & 178447901.0 & 177427065.8 & 179474609.6 \\
400000000.0 & 229488768.0 & 228216803.2 & 230767822.1 \\
500000000.0 & 277279369.5 & 275788078.2 & 278778724.8 \\
. & . & . & .\\
. & . & . & .\\
999900000000.0 & 1838021604.0 & 1682515315.7 & 2007900543.3 \\
\end{tabular}
\end{table}

\newpage

\section{bound\_pop Example}
\label{bound_pop_example}

We examine T-Cell $\beta$ repertoire (TCR$\beta$) sampling
data downloaded from \url{http://mitcr.milaboratory.com/datasets/aging2013/}
and viewed through MiTCR (\url{http://mitcr.milaboratory.com/}).
The first column corresponds to the observed TCR$\beta$ counts.
This can be summarized in a duplicate counts histogram,
easily formed by using the \fn{hist} function in R.

We will first examine the experiment L1\_10\_M9.  The first
ten entries of the duplicate counts histogram is as follows:
\begingroup \fontsize{9pt}{12pt}\selectfont \begin{alltt}
1       433541
2       56447
3       13030
4       4120
5       2841
6       1981
7       1338
8       970
9       814
10      632
\end{alltt}\endgroup

There are 861,156 total observed TCR$\beta$ sequences
and 520,315 distinct TCR$\beta$ sequencing.  We

\begingroup \fontsize{9pt}{12pt}\selectfont \begin{alltt}
$ ./preseq bound_pop -H L1_10_M9_hist.txt -o L1_10_M9_estim_species_richness.txt
\end{alltt}\endgroup

\begin{table}[ht!]
 \fontfamily{pcr}\fontsize{9pt}{12pt}\selectfont
\begin{tabular}{lll}
log\_mean\_estimated\_unobs & log\_lower\_ci & log\_upper\_ci \\
$2976677.3$ & $2862770.9$ & $3095116.0$ \\
\end{tabular}
\end{table}

In quick mode, the output is as follows:
\begin{table}[ht!]
 \fontfamily{pcr}\fontsize{9pt}{12pt}\selectfont
\begin{tabular}{ll}
quadrature\_estimated\_unobs & n\_points \\
$2969646.3$ & $2$ \\
\end{tabular}
\end{table}

\textbf{preseq} estimates that there are at least two million
unobserved TCR$\beta$ sequences in the sample.
This implies that the vast majority of the TCR$\beta$
sequences are unobserved in this sample.




\newpage

\section{\textbf{preseq} Application Examples}

\subsection*{Screening multiple libraries}
\label{sec:multlib}

This section provides a more detailed example using data from
different experiments to illustrate how \textbf{preseq} might be
applied. Because it is important to avoid spending time on low
complexity samples, it is important to decide after observing an
initial experiment whether or not it is beneficial to continue with
sequencing. The data in this example comes from a study (accession
number SRA061610) using single cell sperm cells amplified by Multiple
Annealing and Looping Based Amplification Cycles
(MALBAC)~\cite{lu2012probing} and focuses on three libraries coming
from different experiments from the study (SRX205369, SRX205370,
SRX205372).

These libraries help show what would be considered a relatively poor
library and a relatively good library, as well as compare the
complexity curves obtained from running \fn{c\_curve} and
\fn{lc\_extrap}, to show how \fn{lc\_extrap} would help in the
decision to sequence further. The black diagonal line represents an
ideal library, in which every read is a distinct read (though this
cannot be achieved in reality). The full experiments were down sampled
at 5\% to obtain a mock initial experiment of the libraries, as shown
here, where we have the complexity curves of the initial experiments
generated by \fn{c\_curve}: ~\newline \newline
\begin{figure}[h!]
\centering
\centering{\includegraphics[totalheight = 0.5\textheight, trim=0cm 0cm 0cm 2cm, clip = true]{InitialExperimentComplexityCurves_copy.pdf}}
\caption{Initial observed complexities}
\end{figure}

With such a relatively small amount of reads sequenced, it is hard in
the first stages of a study to guess at whether it is not worth
sequencing a library further, as all three libraries seem to be
relatively good.

This is a comparison of the full experiment complexity curves and the
extrapolated complexity curves created using information from the
initial experiments above as input. The dashed lines indicate the
complexity curves predicted by \fn{lc\_extrap}, and the solid lines
are the expected complexity curves of the full experiments, obtained
using \fn{c\_curve}. Note that the dashed curves follow the solid
curves very closely, only differing slightly towards the end, meaning
\fn{lc\_extrap} gives a good predicted yield curve.  Using this, it is
clear that if the initial experiments were the only available data and
\fn{lc\_extrap} was run, SRX205372 would likely be discarded, as it is
a poor library, and SRX205369 and SRX205370 would probably be used for
further sequencing, as their complexity curves indicate that
sequencing more would yield enough information to justify the costs.
If the researcher were to only want to sequence one library deep, then
SRX205370 would be an obvious choice.
\newline
\newline
\begin{figure}[h!]
\centering
\centering{\includegraphics[totalheight = 0.5\textheight, trim=0cm 0cm 0cm 2cm, clip = true]{FullExperiment_copy.pdf}}
\caption{Estimated versus observed library complexities.}
\end{figure}


\newpage


\subsection*{Saturation of reads and junctions for RNA sequencing experiments}

A recent paper from the Rinn lab~\cite{mercer2011targeted} developed a
targeted capture RNA sequencing protocol to deeply investigate chosen
portions of the transcriptome.  A comparison of the results from a
standard RNA sequencing experiment (RNA-seq; SRA accession SRX061769)
and a targeted capture RNA sequencing experiment (Capture-seq; SRA
accession SRX061768) reveals a startling amount of transcriptional
complexity missed by standard RNA sequencing in the targeted regions.
A large number of rare transcriptional events such as alternative
splices, alternative isoforms, and long non-coding RNAs were newly
identified with the targeted sequencing.

A current vigorous debate exists on whether these rare events are
truly transcriptional events or are merely due to sequencing or
transcriptional noise (see~\cite{van2010most}
and~\cite{clark2011reality}).  We do not seek to address these issues,
but merely to comment on the estimated complexity of rare
transcriptional events in sequencing libraries identified by current
protocols.

We took the two Illumina sequencing libraries
from~\cite{mercer2011targeted} and mapped them according to the
protocol given.  We downsampled $10 \%$ of the library and compared
the estimated library complexities (single end) with the observed
library complexity for both libraries.  We also took the junction
information contained in the file junctions.bed in the Tophat output
folder to estimate the junction complexity.  Since the 5th column
(excluding the first line) is the number of times each distinct
junction is observed, we can simply cut out these values as input for
\fn{lc\_extrap} or \fn{c\_curve} with the flag \fn{-V}.  A simple
command line example follows.
\begin{verbatim}
sed '1d' tophat/junctions.bed | cut -f 5  > junction_vals.txt
./preseq lc_extrap -V junction_vals.txt -o junction_vals_extrap.txt
\end{verbatim}
The output \fn{TOTAL\_READS} column will be in terms of the number of
total junctions (not reads), so scaling by the average number of
junctions per read will give the appropriate scale for plotting on the
x-axis.

\begin{figure}[h!]
\centering{\includegraphics[width=14cm]{compare_RNA_Capture_junction_complexity.pdf}}
\caption{A comparison of complexities of standard RNA-seq and
targeted capture RNA-seq.  Estimated complexities for both cases were
estimated using 10\% of the data.}
\end{figure}

We see from the estimated library that the RNA-seq library is far from
saturated, while it appears that the Capture-seq library may be close.
On the other hand, the junction complexity of both libraries indicates
that the full scope of juctions identified by Tophat is far from
saturated in both libraries.  This indicates that large number of rare
junctions still remain to be identified in the libraries.


\newpage

\subsection*{Comparing coverage for single cell whole genome sequencing library preparations}

We will use \fn{gc\_extrap} to compare coverage profiles across
library preparation protocols.  We use the data from Fu {\em et
  al.}~\cite{fu2015uniform}.  We downloaded SRA accessions ,
SRR1777243, SRR1777245, SRR1777251, SRR1777274, and SRR1777281.  These
correspond to bulk, DOP-PCR, eWGA, MALBAC, and MDA.  We mapped the
reads with bowtie2 to hg19 and ran \fn{gc\_extrap} on the libraries
with default parameters.  To compare the libraries we scaled the
$x$-axis to the total number of sequenced reads by dividing by the
number total bases (TOTAL BASES in verbose mode) and multiplying by
the number of sequenced reads.

\begin{figure}[h!]
\centering{\includegraphics[width=10cm]{comparing_scWGA_coverage.pdf}}
\caption{Coverage comparison of single cell
whole genome sequencing experiments.}
\end{figure}

In our previous paper~\cite{daley2014modeling} we compared coverage
for DOP-PCR, MALBAC, and MDA.  We found that typically DOP-PCR results
in the lowest coverage and MDA the highest.  The results for the
libraries investigated here agree with our previous results.  Notably,
eWGA had the highest estimated coverage.  This indicates that more
investigation of this new library prep is worthwhile to determine if
the observed results hold in general.

\newpage

\subsection*{Estimating and analyzing TCR$\beta$ richness}

In this section we will use \fn{bound\_pop}
to estimate TCR$\beta$ richness from TCR$\beta$
sequencing data taken from Britanova {\em et al.}
\cite{britanova2014age}.  We examined one
of the datasets in section~\ref{bound_pop_example}
but now we investigate all 39 TCR$\beta$ sequencing
experiments.

The average number of observations, or sampled
TCR$\beta$ sequences, was 992,359 with a
range of  $($861,156, 999,024$)$
indicating that all experiments were sampled to a similar
depth.
In contrast, the range of the number of distinct TCR$\beta$
sequences observed was quite large with a minimum of
133,464, a maximum of 772,223, and a mean of 461,797.

The observed TCR$\beta$ richness is obviously biased so
we used \fn{bound\_pop} to estimate the total richness.
The estimated TCR$\beta$ richness ranges from a minimum
of 823,223 to a maximum of 8,945,797 with a mean
of 2,969,305.

Note that any statistical test for relationship will be liberal
since the estimation procedure adds extra variance that is not
apparent from the estimated values~\cite{willis2015inference}.
A standard test will tend to reject more often when the
null hypothesis is false and will tend to accept less often
when the null hypothesis is true.
Therefore we can use standard tests to determine
where to investigate further.

First there appears to be no difference in
estimated TCR$\beta$ richness between the sexes
of the participants (mean of 3,296,501 from 19 females
versus 2,658,468 from 20 males).
We next investigate the relationship between the age
of the participant and the estimated TCR$\beta$ richness,
a topic investigated by Britanova {\em et al.}



 \begin{figure}[h!]
\centering{\includegraphics[width=8cm]{TCR_richness_vs_age_lm.pdf}}
\caption{Estimated TCR$\beta$ richness versus age and the
least square fit.  $R^{2} = 0.452$, intercept $ = 5,117,862$, slope $= -43,802$.
}
\end{figure}

We see that there is a strong negative relationship between the
estimated species richness and age.  Interestingly we see a lack
of homoscedasticity in the data, as the data shows a noticable
decrease in variation as a function of age.
Clearly further investigation into the relationship between
immune repertoire and age will be of interest.



\newpage


\section{FAQ}

\Que{When compiling the \textbf{preseq} binary, I receive the error

\fn{fatal error: gsl/gsl\_cdf.h: No such file or directory
}
}

\Ans{The default location of the GSL library will be in '\fn{/usr/local/include/gsl}'. Open the Makefile and append "\fn{-I /usr/local/include}" after \fn{CXX = g++}.  You may be receiving this error because the GSL library is not installed on the default search paths of your compiler, and you will need to specify the location.  }




\Que{When compiling the \textbf{preseq} binary, I receive the error

\fn{Undefined symbols for architecture x86\_64: ~\\
\tab"\_packInt16", referenced from:~\\
\tab\tab\_deflate\_block in bgzf.o~\\
 \tab"\_packInt32", referenced from:~\\
 \tab\tab\_deflate\_block in bgzf.o~\\
 \tab "\_unpackInt16", referenced from:~\\
   \tab\tab   \_bgzf\_read\_block in bgzf.o~\\
    \tab\tab  \_check\_header in bgzf.o~\\
}
}

\Ans{
Go to the SAMTools directory and open the file bgzf.c. Find the functions \fn{packInt16}, \fn{unpackInt16}, and \fn{packInt32}. Comment out the "inline" before each function name.
}



\Que{I compile the \textbf{preseq} binary but receive the error

\fn{terminate called after throwing an instance of 'std::string'}
}

\Ans{This error is typically called because either the flag -B was not included to
specify bam input or because the linking to SAMTools was not included when
compiling \textbf{preseq}.  To ensure that the linking was done properly, check for the flag
\fn{-DHAVE\_SAMTOOLS}.}

\Que{When running \fn{lc\_extrap}, I receive the error

\fn{ERROR:  too many iterations, poor sample}

}

\Ans{
Most commonly this is due to the presence of
defects in the approximation which cause the estimates to be unstable.  Setting the step
size larger (with the flag \fn{-s}) will help to avoid the defects.  The default step size is 1M reads or
0.05\% of the input sample size rounded up to the nearest million, whichever is larger.
A consequence of this action will be a reduction in the observed smoothness of the curve.
}


\Que{When running \fn{lc\_extrap}, I receive the error

\fn{sample not sufficiently deep or duplicates removed}
}

\Ans{There may be two causes for this, either duplicates have been removed and
every observed read is distinct or there is not sufficient variation in the library for
\fn{lc\_extrap} to run.

The information required by \fn{lc\_extrap} is essentially the number of times each
distinct read was observed, which we call the duplicate counts.
Without sufficient variation in the duplicate counts we cannot extrapolate the
complexity of the library.  We have set the minimum required max duplicate count
(the largest number of times any read has been observed) to 4.
If the input library does not satisfy this, then either a parametric model such
as a Poisson or Negative Binomial may be appropriate or deeper
sequencing may be required.}

\Que{When running \fn{lc\_extrap}, I receive the error

\fn{Library expected to saturate in doubling of size, unable to extrapolate}
}

\Ans{
A simple two-fold extrapolation using the Good-Toulmin
power series, which is within the radius of convergence and
therefore rational function approximation is not needed,
is performed to ensure that the sample is not overly saturated.
If the Good-Toulmin formula is negative, this indicates that the
library will likely completely saturate by doubling the experiment
size and so extrapolation is not needed.  Often this will occur
if the number of reads observed twice ($n_{2}$) is greater than the number
of reads observed once ($n_{1}$).  In this case one can use
simple estimators like Chao's~\cite{chao1987estimating} ($n_{1}^2 / 2 n_{2}$)
or Zelterman's~\cite{zelterman1988robust}
($1 / (exp(2 n_{2}/ n_{1}) - 1)$) can be used to estimate
the number of remaining in the library.
}

\vspace{5mm}
If none of these solutions worked, please email us at
\href{mailto:tdaley@usc.edu}{\nolinkurl{tdaley@usc.edu}}
and please include the standard output from running \textbf{preseq} in
verbose mode (specifically the duplicate counts histogram) so
that we can look into the problem and rectify problems in future
versions.  Also, feel free to email us with any other questions or
concerns.
The \textbf{preseq} software is still under development so we would appreciate any
advice, comments, or notification of any possible bugs. Thanks!

\newpage

\bibliographystyle{plain}
\bibliography{biblio}





\end{document}
